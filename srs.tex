\documentclass[a4paper,12pt]{article}

\usepackage[margin=2cm]{geometry}
\usepackage{graphicx}
\usepackage{amsmath}
\usepackage{array}
\usepackage{hyperref}
\usepackage[all]{hypcap}
\usepackage{listings}
\lstdefinestyle{TerminalStyle}{
  language=bash,
  basicstyle=\small\sffamily,
  numbers=left,
  numberstyle=\tiny,
  numbersep=3pt,
  frame=tb,
  columns=fullflexible,
  linewidth=0.9\linewidth,
  xleftmargin=0.1\linewidth
}
\lstdefinestyle{HtmlStyle}{
  language=html,
  basicstyle=\small\sffamily,
  numbers=left,
  numberstyle=\tiny,
  numbersep=3pt,
  frame=tb,
  columns=fullflexible,
  linewidth=0.9\linewidth,
  xleftmargin=0.1\linewidth
}
\lstdefinestyle{OutputStyle}{
  language=html,
  basicstyle=\small\sffamily,
  frame=tb,
  columns=fullflexible,
  linewidth=0.9\linewidth,
  xleftmargin=0.1\linewidth
}

\setlength{\parindent}{0pt}
\setlength{\parskip}{1ex plus 0.5ex minus 0.2ex}

\title{\includegraphics[width=12cm]{Eeufeeslogo.jpg} \\
       Department of Computer Science \\
       University of Pretoria \\
       \vspace{0.5cm}
       COS730 Software Engineering \\
       \vspace{0.5cm}
       \begin{large} \textbf{Software Requirement Specifications}\end{large}}

\date{} 
\author{	JM Bondjobo		13232852 		\\
		Martha Mohlala		10353403
		 \\
		 KJ Muranga         13278012        \\
}

\begin{document}
\maketitle
\thispagestyle{empty}
\clearpage

\newpage
\pagenumbering{roman}
\thispagestyle{empty}
\tableofcontents
\clearpage

\newpage
\pagenumbering{arabic}

\section {Introduction}
This section gives a scope description of the benchmark service system. Also, the purpose for this document is described and a list of abbreviations and definitions is provided.
\subsection{Purpose}
The goal of this document is to give a detailed description of the  benchmarking service. It will explain the purpose and features of the system, the interfaces of the system, what the system will do and the constraints under which it must operate.


\subsection{Scope}
The benchmarking service will accept a user's source code and measure its performance in terms of CPU time, elapsed time, memory usage, power consumption, heat generation, and others.The user's source code could be either an algorithm or a data structure, and the system should give the user a choice of which programming languages their source code can be implemented in. The service should be provided by executing the requested benchmarks on isolated machines where the side-effects that are not a concern of the specified benchmark are minimized.

The system will be used as an integral part of research related to the development of new algorithms and techniques as well as the refinement and optimisation of existing operations. It can also be used to as a teaching tool for students to review the notions of space and time complexity.



\subsection{Definitions, Acronyms and Abbreviations}
\begin{center}
\begin{tabular}{ |l|l{\textwidth}| } 
\hline
Term & Definition \\ 
\hline
User & Individual who uses the benchmarking system. \\ 
\hline
Software Requirements Specification & A document that completely describes all of the functions of a proposed system and the constraints under which it must operate. For example, this document. \\ 
\hline
Benchmark & Evaluate (something) by comparison with a standard.
\hline
\end{tabular}
\end{center}

\subsection{References}
IEEE. IEEE Std 830-1998 IEEE Recommended Practice for Software Requirements Specifications. IEEE Computer Society, 1998

\subsection{Overview}
The remainder of this document consists of two chapters. The second chapter, the Overall Description section, of this document gives a description of the system's functionality. It describes the informal requirements and is used to establish a context for the technical requirements specification in the next chapter.

The third chapter, Specific Requirements section, of this document gives a detailed description of the benchmarking service. It describes in technical terms the details of the functionality of the system.

\section{Overall Description}



\subsection{Product Perspective}

\subsubsection{System Interfaces}
The system will provide a web interface for users to request and specify the benchmarking services they need. The system interface is depicted as below:\\
\includegraphics[width=12cm]{SI.jpeg}
\subsubsection{User Interfaces}
The system will have a user interface where s/he is interacting with the system by requesting some services and getting a response from the system done at the backend. This is the front end part of the system that the client/user will be seing. It must follow basic Windows style and functionality conventions. The interface will have a menu where the services can be requested, a log button on the menu to see what services were done on the pc and by who and a quick access button to terminate the benchmarking process if something goes wrong.
\subsubsection{Hardware Interfaces}
The system will run a computer that has is Windows 7 and higher and that has at least 100 Mb of free space on the hard drive and 1Gb of Ram. That pc used must be an isolated machine where the benchmarking service will be done in order to reduce/ minimize the side effects of others programs running simultaneously. The system does not write information directly to the user's computer. But instead uses a database which is located on a network server. The user's computer transfers and receives data using basic networking protocols like HTTP. All systems' information is stored in the server's database which stores the data on the server's disk reason being if the system runs out of memory or fails during benchmarking the previously gathered information remains on the server's database.
\subsubsection{Softwares Interfaces}
The system requires a a properly configuered version of Windows 7 or later to run the application. The system's server can either be Windows or Linux but must have a MySQL databse properly installed and configured. The software interface will be web based, therefore must have Microsoft .NET Framework 1.1 or greater installed and for the designing of the benchmarking that can be done using Adobe XD.

\subsubsection{Communications Interfaces}
The benchmarking system will have a network server that is web-based. The server exists to retrieve information from the database
and compute the services in terms of the benchmark metrics provided. The web interface will also query the database for the historical computations of the benchmark metrics. The HTTP server will be used to send the requests from the web interface to
the benchmarking system. The web application will support all types of browsers.
\subsubsection{Memory}
The system will not use memory less than 100Mb of the hard drive and 1Gb of RAM as described on the hardware interface.
\subsubsection{Operations}
The user will initiate the request for benchmarking services and wait for response. The response might take a while as the system is required to do computation and comparison between two inputs. The user will have an option of inputting algorithm or data structure and the input will be validated to check if it is algorithm or data structure. The system will also allow the user to store all the computation and comparison results to the database as historical data to avoid computing the same request many times,and also for recovery. 
\subsubsection{Site Adaptation Requirements}

\subsection{Product Functions}

\subsection{User Characteristics}

\subsection{Constraints}

\subsection{Assumptions and Dependencies}

\section{Specific Requirements}

\subsection{External Interface Requirements}

\subsection{Functional Requirements}

\subsection{Performance Requirements}

\subsection{Design Constraints}

\subsection{Software System Attributes}

\subsection{Other Requirements}

\newpage
\clearpage
\addcontentsline{toc}{section}{References}

\end{document}
